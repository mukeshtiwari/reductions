We define map expressions as 
\[
r, r' := b \mid r \times r' | r + r' | r' ; r | r \circ r'
\]
where $b$ is a basic reduction. 
\[
\begin{array}{rcl}
  ( r \times r') (a, b) & \equiv & (r(a),\ r'(b)) \\
  ( r + r') (inl(a)) & \equiv & inl (r (a)) \\
  ( r + r') (inr(b)) & \equiv & inl (r' (b)) \\
  ( r' ; r)(a) & \equiv & r (r'(a)) \\    
  ( r \circ r')(a) & \equiv & r (r'(a)) \\  
\end{array}   
\] 

We show that the following rules obtain. 
\[
\begin{array}{c}
  \begin{array}{c}
  r \propto (S,\ \bullet) \quad \quad \quad r' \propto (T,\ \diamond) \\ \hline
  (r \times r') \propto (S \times T,\ \bullet \times \diamond)
  \end{array}     
  \\
  \\
  \begin{array}{c}
  r \propto (S,\ \bullet) \quad \quad \quad r' \propto (T,\ \diamond) \\ \hline
  (r + r') \propto (S + T,\ \bullet + \diamond)
  \end{array}     
  \\
  \\
  \begin{array}{c}
  r' \propto (S,\ \bullet) \quad \quad \quad r \propto (S_{r'},\ \bullet_{r'}) \\ \hline
  (r' ; r) \propto (S,\ \bullet)
  \end{array}     
  \\
  \\
  \begin{array}{c}
  r' \propto (S,\ \bullet) \quad \quad \quad r \propto (S,\ \bullet) \quad \quad \quad \forall s \in S, r(r'(s)) = r'(r(s))\\ \hline
  (r \circ r') \propto (S,\ \bullet)
  \end{array}     
\end{array}   
\] 
The first three rules are easy to prove.
Therefore, let us focus on look at map composition $r \circ r'$. 
We must have
\[
\begin{array}{rcl}
  S_{r \circ r'} & = & \{s \in S\mid (r \circ r')(s) = s\} \\
               & = & \{s \in S\mid r (r' s) = s\} 
\end{array}   
\] 
and 
\[
\begin{array}{rcl}
\propname{RIP}(S,\ r \circ r')    
    & \leftrightarrow 
    & \forall x \in S,\ r(r'(r(r'(x)))) = r(r'(x))\\
\propname{RLC}((S,\ \bullet),\ r \circ r')
    & \leftrightarrow 
    & \forall x, y \in S,\ r(r'((r(r'(x)) \bullet y)) = r(r'(x \bullet y))\\ 
\propname{RRC}((S,\ \bullet),\ r \circ r')
    & \leftrightarrow 
    & \forall x, y \in S,\ r(r'(x \bullet r(r'(y)))) = r(r'(x \bullet y))\\ 
\end{array}   
\]
It seems very difficult to come up with a general rule for this construction
other than imposing all of the complex conditions. 
However, we can identify a special case that is easy to work with.
Suppose that $r$ and $r'$ \emph{commute} as maps. 
That is, 
\[
r \circ r' = r' \circ r. 
\]
In addition, assyme that
$r$ and $r'$ are reductions over $(S, \bullet)$. 
Then, for 
$\propname{RIP}(S,\ r \circ r')$
we have
\[
\begin{array}{rcl}
(r \circ r')((r \circ r')(s))
  & =
  & r(r'(r(r'(x)))) \\
  & =  
  & r'(r'(r(r(x)))) \\ 
  & = 
    & r'(r(r(x))) \\
    & = 
    & r(r'(x))\\
    & =   
    & (r \circ r')(s))
\end{array}   
\] 
and for 
$\propname{RLC}((S,\ \bullet),\ r \circ r')$
we have
\[
\begin{array}{rcl}
  (r \circ r')((r \circ r')(x) \bullet y)
  & =  
  & r(r'((r(r'(x)) \bullet y)) \\ 
  & =
  & r(r'((r'(r(x)) \bullet y)) \\
  & =
  & r(r'(r(x) \bullet y)) \\
  & =
  & r'(r(r(x) \bullet y)) \\
  & =
  & r'(r(x \bullet y)) \\
  & =
  & r(r'(x \bullet y)) \\
  & =  
  & (r \circ r')(x \bullet y)
\end{array}   
\] 
A similar proof holds for
$\propname{RRC}((S,\ \bullet),\ r \circ r')$. 

\begin{theorem}
  If $r$ and $r'$ are both reductions over $(S,\ \bullet)$ and
  $r$ and $r'$ commute, then 
  $r \circ r'$ is a reduction over $(S,\ \bullet)$. 
\end{theorem}

Note that the semantics of $r' ; r$ and $r \circ r'$ are the same.
That is, 
\[
\begin{array}{rcl}
  ( r' ; r)(a) & \equiv & r (r'(a)) \\    
  ( r \circ r')(a) & \equiv & r (r'(a)) \\  
\end{array}   
\] 
What is different is the pre-conditions needed in our inference rules. 
However, an interesting question to ask this:
When are
$r' ; r$ and $r \circ r'$
equal \emph{as reductions} on $(S,\ \bullet)$? 


Suppose that  $r$ and $r'$ are reduction over $(S,\ \bullet)$ and
$r$ and $r'$ commute. 
Then $r$ is a reduction over $(S_{r'},\ \bullet_{r'})$.
To see this, note that
$\propname{RIP}(S_{r'},\ r)$
means $\forall s \in S_{r'} r(r(s)) = s$.
This holds since we are assuming
$\propname{RIP}(S,\ r)$ and 
and $S_r\subseteq S$. 
Next, consider 
$\propname{RLC}((S_{r'},\ \bullet_{r'}),\ r)$.
Suppose $x, y\in S$, 
$r'(x) = x$,  and
$r'(y) = y$.
Then 
\[
\begin{array}{rcl}
  r(r(x) \bullet_{r'} y)
  & =
  r(r'(r(x) \bullet y)) \\ 
  & =  
  & r(r'((r(r'(x)) \bullet y)) \\ 
  & =
  & r(r'((r'(r(x)) \bullet y)) \\
  & =
  & r(r'(r(x) \bullet y)) \\
  & =
  & r'(r(r(x) \bullet y)) \\
  & =
  & r'(r(x \bullet y)) \\
  & =
  & r(r'(x \bullet y)) \\
  & =  
  & r(x \bullet_{r'} y)
\end{array}   
\] 
A similar argument holds for
$\propname{RRC}((S_{r'},\ \bullet_{r'}),\ r)$.

\begin{theorem}
  If $r$ and $r'$ are both reductions over $(S,\ \bullet)$ and
  $r$ and $r'$ commute, then
  $r \circ r'$ and $r' ; r$ are identical reductions over $(S,\ \bullet)$. 
\end{theorem}

What about the case where $r$ and $r'$ do not commute?
We leave this as on open question. 
