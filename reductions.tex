
Algebraic reductions for semigroups were fist defined
by Wongseelashote~\cite{Wongseelashote79}.

Let $r$ be a map $r \in S \rightarrow S$  and
$(S,\ \bullet)$ a semigroup.
Define 
\[
\begin{array}{rcl}
  \mathrm{reduce}((S,\ \bullet),\ r) & \equiv & (S_r,\ \bullet_r) \\
\end{array}  
\]
where
\[
\begin{array}{rcl} 
S_r           & \equiv & \{s \in S\mid r(s) = s\} \\
x \bullet_r y & \equiv & r(x \bullet y)
\end{array}  
\]
If $(S_r, \ \bullet_r)$ is a semigroup, then
we say that $r$ is a \emph{semigroup reduction}.

Wongseelashote~\cite{Wongseelashote79} defined sufficient
conditions on $r$ that ensure it is a reduction: 
\[
\begin{array}{rcl}
\propname{RIP}(S,\ r)    
    & \equiv 
    & \forall x \in S,\ r(r(x)) = r(x)\\
\propname{RLC}((S,\ \bullet),\ r)
    & \equiv 
    & \forall x, y \in S,\ r(r(x) \bullet y) = r(x \bullet y)\\ 
\propname{RRC}((S,\ \bullet),\ r)
    & \equiv 
    & \forall x, y \in S,\ r(x \bullet r(y)) = r(x \bullet y)\\ 
\end{array}   
\]
We will write $r \propto (S,\ \bullet)$ to mean that
$r$ is a reduction for $(S,\ \bullet)$. 


In a similar way, if
$(S,\ \oplus,\ \otimes)$ is a semiring, then
\[
\begin{array}{rcl}
  \mathrm{reduce}((S,\ \oplus,\ \otimes),\ r) & \equiv & (S_r,\ \oplus_r,\ \otimes_r). \\
\end{array}  
\]
If this results in a semiring, then we will call $r$ a
a \emph{semiring reduction}.
The same sufficient conditions can be used in this case. 
